\documentclass[a4paper]{letter}
\usepackage[latin1]{inputenc}
\usepackage{ngerman}
\usepackage{color}
\usepackage{graphicx}
\usepackage{amssymb}
\usepackage{fullpage}
\usepackage{scrextend}
\usepackage{amsmath}
\usepackage{amssymb}
\usepackage{amsthm}

% \setlength{\textwidth}{14cm}
% \setlength{\oddsidemargin}{0mm}
% \setlength{\evensidemargin}{0mm}
% \setlength{\unitlength}{1mm}
% \setlength{\textheight}{22cm}
% \setlength{\voffset}{0cm}

\begin{document}

\begin{center}
	\underline{
		\textbf{
			\large{
				Folien zum Verfahren der konjugierten Gradienten
			}
		}
	}
\end{center}

\begin{center}
Michael Bauer, 11. November 2013
\end{center}

\parskip 20pt

\underline{Richtung des steilsten Abstiegs}
\\\\Sei f wie in (1.1). Die \underline{Richtung des steilsten Abstiegs} von f an der Stelle x, d.h. $s\in\mathbb{R}^{n}$ so, dass die Richtungsableitung
$$\frac d {dt} f(x+t\frac s {\|s\|_{2}})|_{t=0} = (\nabla f(x))^{T} (\frac s {\|s\|_{2}}) \hspace{10mm}(1.2)$$
minimal ist, wird durch $s = -\nabla f(x) = b - Ax$ gegeben.

\underline{Beweis:}
\\Aus den Eigenschaften des Skalarproduktes wissen wir, dass $\langle x, y \rangle$ minimal wird genau dann, wenn $y = -x$. Da $\nabla f(x) = Ax - b$ und f\"ur festes $x$ muss $s$ in $\langle {\nabla f(x)}, {\frac s {\|s\|_{2}}} \rangle$ zu $\nabla f(x)$ entgegengesetzte Richtung haben, also $s = -\nabla f(x) \Rightarrow$ Beh.$\hfill \blacksquare$

\parskip 20pt

\underline{Projektionssatz aus Numerik 1}
\\\\F\"ur $U \subset V$, $U$ sei ein n-dim. Teilraum von $V$ und $\phi_{j}$ eine ONB. Dann existiert ein eindeutiges $u^{*} \in U$, welches $\|u^{*} - v\| = \underset{u \in U}{\min} \|u - v\|$ erf\"ullt. F\"ur jedes $v \in V$ wird dieses Problem durch
$$P_{U}(v) := \sum_{j=1}^{n} \langle v, \phi_{j} \rangle \phi_{j}$$
gel\"ost. $P_{U}(v)$ ist die \underline{orthogonale Projektion bzgl. $\langle \cdot, \cdot \rangle$}.

\parskip 20pt

\underline{Krylovraum}
\begin{addmargin}[20mm]{20mm}
$\mathcal{K}_{\mathit{k}}(r, A) := span\{r, Ar, ..., A^{k-1}r\} \hspace{20mm} mit \hfill k \ge 1$
\\$\mathcal{K}_{\mathit{k}}(r, A) := \{0\} \hspace{48mm} mit \hfill k = 0$
\end{addmargin}
hei{\ss}t Krylovraum zur Matrix $A$ und zum Vektor $r$.
\\Iterative Krylovraum-Methoden zur L\"osung eines GLS $Ax = b$ mit $A \in \mathbb{R}^{n \times n}$ verlangen $x^k \in x^{0} + \mathcal{K}_{\mathit{k}}(r^{0}, A)$ und $x^{n} = x^{*}$, wobei $r^{0} = Ax^{0} - b$.

\end{document}